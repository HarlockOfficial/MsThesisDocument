\documentclass{unicam_thesis}
\usepackage[utf8]{inputenc}
\usepackage[T1]{fontenc}
\usepackage[table]{xcolor}
\usepackage{listings}
\usepackage{spverbatim}
\usepackage[Sonny]{fncychap}          %Sonny, Lenny, Conny, Bjornstrup
\usepackage{epigraph}
\usepackage{acro}
\usepackage[nonumberlist]{glossaries}
%\usepackage[tracking]{microtype}
\usepackage{lmodern}
\usepackage{enumitem}
\usepackage{booktabs}
\usepackage{tabularx}
\usepackage{varioref}
\usepackage{comment}
%\usepackage{fontspec}
\setlength{\marginparwidth}{2cm} % REmove later with todonotes
\usepackage{todonotes} % TODO Notes

%%%% FONTS %%%%
% Palatino for text
%\usepackage[sc,osf]{mathpazo}   % With old-style figures and real smallcaps.
%\linespread{1.025}              % Palatino leads a little more leading
%\usepackage{beton}
%\DeclareFontSeriesDefault[rm]{bf}{sbc}
%\setmainfont{Corbel}
\usepackage{concmath}
\usepackage{libertine}
\usepackage{xcolor}

\lstset{
    basicstyle=\ttfamily,
    escapeinside={§}{§}, % define the escape characters
}
% Euler for math and numbers
%\usepackage[euler-digits,small]{eulervm}
%\AtBeginDocument{\renewcommand{\hbar}{\hslash}}

%\renewcommand\thesection{\arabic{section}}


% alternatively use beton for text and concmath for math
% beton needs microtype disabled and needs additional commands for bold
%%%%%%%%%%%%%%%%%%%%%%%%%%%%%%%%%%%%%%%%%%%%%%%%%%

% definition environment
%\theoremstyle{definition}
%\newtheorem{definition}{Definition}[section]

%allows numbering of paragraphs
%\renewcommand{\theparagraph}{\S\arabic{paragraph}} 
%\setcounter{secnumdepth}{4} 

%\rowcolors{2}{gray!25}{} %table colours

%\setlength\marginparwidth{2cm} %used to make todos readable, remove on publication


%%% narsese listings %%%
% \definecolor{bluekeywords}{rgb}{0.13,0.13,1}
% \definecolor{greencomments}{rgb}{0,0.5,0}
% \definecolor{redstrings}{rgb}{0.9,0,0}

% \lstdefinelanguage{narsese}{
%     otherkeywords = {},
%     morekeywords = {}, % blue keywords here
%     morekeywords = [2]{Answer},  % red keywords here
%     morecomment = [s][\color{blue}]{\%}{\%},
%     sensitive=false,
% }

% \lstset{
%   language=narsese,
%   showspaces=false,
%   showtabs=false,
%   breaklines=true,
%   showstringspaces=false,
%   breakatwhitespace=true,
%   escapeinside={(*@}{@*)},
%   commentstyle=\color{greencomments},
%   keywordstyle=\color{bluekeywords},
%   keywordstyle = [2]{\color{red}},
%   stringstyle=\color{redstrings},
%   basicstyle=\ttfamily,
%   %numbers=left,
%   frame=single
% }


% %%% acronyms %%%
\acsetup{first-style=short}
\DeclareAcronym{VM}{
   short = VM,
   long  = Virtual Machine,
   tag = abbrev
 }
% %%% glossaries %%%
% \makeglossaries

% \newglossaryentry{World}
% {
%     name=World,
%     description={A \textbf{world} $W$ is a interactive system consisting of a set of variables $V$, dynamics functions $F$, an initial state $S_0$, domains $D$ of possible clusters of particular constraints on their values, and a set of relations between the variables $R$: $W = \langle V, F, S_0, D, R \rangle$. The variables $V = \{ v_1, v_2, ..., v_{\|V\|}\}$ represent anything that may change or hold a particular value in the world. The dynamics functions act as the laws of nature in the world and as a whole can be seen as an automatically executed function that periodically or continually evolves the world's current state into the next: $S_{t + \delta} = F(S_t)$. In practice it is useful to the decompose the dynamics into a set of transition functions: $F = \{f_1, f_2, ..., f_n\}$ where $f_i: S^- \rightarrow S^-$ and $S^-$ is a partial state. The domains $d_v \in D$ specify which values each variable $v$ can take, and for physical domains these are usually subsets of real numbers. The relations are Boolean functions over variables that hold true in any state the world will ever find itself in. If the world is a closed system with no outside interference, the domains and relations are implicitly fully determined by the dynamics functions and the initial state. In an open system where changes can be caused externally, instead, the explicit definition of domains and invariant relations can restrict the range of possible interactions \citep{whytask}}
% }

% \newglossaryentry{Environment}
% {
%     name=Environment,
%     description={An \textbf{environment} is a view of a \textbf{world}. The body of an agent is considered to be part of it. \citep{whytask}}
% }

% \newglossaryentry{State}
% {
%     name=State,
%     description={A \textbf{state} can be \textit{concrete} or \textit{partial}. A \textit{concrete} state $S$ is a value assignment to all of the variables in a \textbf{task-environment}: $S = \bigcup_{v \in V}\{\langle v, x_v \mid x_v \in d_v \rangle\}$ A \textit{partial} state $S^-$ only assigns values to a subset of the variables. When considering real variables partial states can be represented using error bounds: $S^- = \bigcup_{v \in V^-} \{\langle v, x_l, x_u \mid x_l < x_u \land (x_l, x_u) \subseteq d_v \rangle \}$; this way a partial state covers a set of concrete states. A state is valid if and only if all invariant relations hold: $\text{valid}(S) \iff \forall_{r \in R} r(S) $. In practice the presence of noise and the partial observability of variables makes the use of partial states more practical than concrete states, therefore by state is always meant a partial state unless otherwise noted \citep{whytask}}
% }

% \newglossaryentry{Agent}
% {
%     name=Agent,
%     description={An \textbf{agent} is an embodied system consisting of a controller (the mind) and a body. The body is the agent's interface to the \textbf{world} which allows the perception of the external \textbf{environment}, through the flow of data from the body's sensors to the controller, and the execution of atomic actions, by means of the commands sent from the controller to the body's actuators. The body contains two lists of variables that the controller can read and write to: $B = \langle V_S, V_A \rangle$. Since the body is a physical entity, its sensors and actuators are physical objects in the world as well and are treated as such \citep{whytask}}
% }

% \newglossaryentry{Goal State}
% {
%     name=Goal,
%     description={A \textbf{goal} state is a desirable, possibly partial, \textbf{state} that the agent should reach \citep{whytask}}
% }

% \newglossaryentry{Failure State}
% {
%     name=Failure,
%     description={A \textbf{failure} state is an undesirable, possibly partial, \textbf{state} that the agent should avoid \citep{whytask}}
% }

% \newglossaryentry{Problem}
% {
%     name=Problem,
%     description={A \textbf{problem} can be \textit{atomic} or \textit{compound}. An atomic problem is specified by an initial \textbf{state}, \textbf{goal} states and \textbf{failure} states. A compound problem can be created by composition of atomic problems using operators such as conjunction, disjunction and negation. A problem for which a solution is known to exist is called a closed problem \citep{whytask}}
% }

% \newglossaryentry{Solution}
% {
%     name=Solution,
%     description={A \textbf{solution} is a sequence of atomic actions that results in a path through the \textbf{state} space that reaches all of the \textbf{goal} states and none of the \textbf{failure} states \citep{whytask}}
% }

% \newglossaryentry{Task}
% {
%     name=Task,
%     description={A \textbf{task} is a problem assigned to an \textbf{agent}, $T = \langle \mathcal{S}_0, \mathcal{G}_{top}, \mathcal{G}_{sub}, G^-, B, t_{go}, t_{stop}, I \rangle$, where $S_0$ is the set of permissible initial states, $\mathcal{G}_{top}$ is the task's set of top-level goals, $\mathcal{G}_{sub}$ is the set of given sub-goals, $G^-$ is its set of constraints, $B$ is a controller's body, and $t$ refers to the permissible start and stop times of the task. An \textit{assigned} task will have all its variables bound and reference an agency that is to perform it (accepted assignments having their own timestamp $t_{assign}$). This assignment includes the manner in which the task is communicated to the agent, for example if the agent is given a description of the task a priori, receives additional hints or if it only gets incremental reinforcement signals as certain \textbf{states} are reached. A task is performed successfully when the \textbf{world}'s history contains a path of states that solved the problem \citep{whytask}}
% }

% \newglossaryentry{Task-Environment}
% {
%     name=Task-Environment,
%     description={By \textbf{task-environment} is meant the tuple of a \textbf{task} and the \textbf{environment} in which it is to be performed. The separation of a task from its environment is not always clear and somewhat arbitrary, therefore the term task-environment is used to encompass all the relevant aspects of both \citep{whytask, EvaluatingUnd}}
% }

% \newglossaryentry{Phenomenon}
% {
%     name=Phenomenon,
%     description={A \textbf{phenomenon} (process, state of affairs, occurrence) $\Phi$, where $W$ is the \textbf{world} and $\Phi \subset W$, is composed of a set of elements $\{\phi_1, \phi_2, ..., \phi_n \in \Phi \}$ of various kinds including relations $\mathfrak{R}_\Phi$ that couple elements of $\Phi$ with each other and with those of other phenomena. The elements that a phenomenon is made up of can be any sub-division of $\Phi$, including sub-structures, causal relations, whole-part relations and so on. The relations $\mathfrak{R}_\Phi \subseteq 2^W \times 2^W$ that extend to other phenomena identify the phenomenon's \textit{context}. The set of relations can be partitioned in \textit{inward facing} relations $\mathfrak{R}_\Phi^{in}=\mathfrak{R}_\Phi \cap (2^\Phi \times 2^\Phi)$ and \textit{outward facing} relations $\mathfrak{R}_\Phi^{out}=\mathfrak{R}_\Phi \setminus \mathfrak{R}_\Phi^{in}$ \citep{AboutUnd}}
% }

% \newglossaryentry{Problem space}{    
%     name=Problem space,
%     description={The problem space is the set of all valid states of the task}
%     }

% \newglossaryentry{Solution space}{    
%     name=Solution space,
%     description={The solution space is the subset of the problem space defined by the task's goals and constraints, made up of all the solution states reachable from any initial state of the task}
%     }


%%%% notation %%%%%
% \usepackage{nomencl}
% \makenomenclature
% \nomenclature[01]{$X$}{Random variable}
% \nomenclature[02]{$x$}{Value of a random variable}
% \nomenclature[03]{$\mathbf{X}$}{Vector}
% \nomenclature[04]{$P(Y\mid X = x)$}{Conditional distribution}
% \nomenclature[05]{$P(Y\mid \text{do}(X = x))$}{Intervention distribution}
% \nomenclature[06]{$P(Y\mid Z = \hat{z}, X = \hat{x}, \text{do}(X = x))$}{Counterfactual distribution}

%%%%%%%%%%%%%%%%%%%%%%%%%%%%
% TESI DATI FRONTESPIZIO
%%%%%%%%%%%%%%%%%%%%%%%%%%%%

\title{EEG - Motor Imagery}

\universityb{Reykjav\'ik University}%
\schoolb{Department of \\ Computer Science}%

\university{University of Camerino}%
\school{School of Science \\ and Technology}%
\course{Master of Science in Computer Science}%

\matricola{122435}%

\author{Francesco Moschella}%
\advisor{Prof. Dr. Michele Loreti}%
\advisortwo{Prof. Dr. Hannes Viljamson H\"ogni}%
\coadvisor{Prof. Dr. Michela Quadrini}%
\coadvisortwo{Dr. Nicola Del Giudice}%

\academicyear{2024/2025}%

%%%%%%%%%%%%%%%%%%%%%%%%%%%%
% FINE DATI FRONTESPIZIO
%%%%%%%%%%%%%%%%%%%%%%%%%%%%



\graphicspath{{Screenshot/},{Immagini/},{API/},{Source/}}
\bibliography{biblio.bib}

\begin{document}


\maketitle





\tableofcontents
% \lstlistoflistings
% \listoffigures
% \listoftables
% \printacronyms[include=abbrev, name={Acronyms}, heading=chapter*, display=all]
% \glsaddall
\begin{abstract}{
    Nowadays, enhancing the quality of life for individuals with reduced mobility or physical disabilities is a crucial challenge.
    In recent years, new technologies and techniques, such as Human-Computer Interfaces and, more specifically, Brain-Computer Interfaces have emerged as viable solutions.
    The research has focused on developing applications and devices that allow people to complete everyday tasks autonomously or with minimal need for external caregivers.
    This thesis focuses on brain-computer interfaces and tries to solve one major problem of these kinds of technologies: the time gap between the user action intention and the actual application of it.
    
    We present a quasi-real-time ElectroEncephaloGraphy Motor Imagery classification system, a pipeline to integrate the classification system and seamlessly control external devices or applications, and a data augmentation system that allows rapid and efficient testing.
    We will also present a virtual environment that we used to test the pipeline.
    
    We tested the system with an EEG cap and volunteers to verify the network and the pipeline.
    We found the network to be accurate and resource-efficient, and the pipeline fast and flawless, with an average of 650~ms between the user movement intention and the application in the virtual environment.
    This is an improvement over the average values found during the literature review, which report the need for an EEG recording between 4 and 8 seconds that has to be processed, classified, and sent to the controlled device.

    This thesis is an initial work and a feasibility study towards real-time brain-controlled devices and applications.
    It can also enhance the quality of life for people with reduced mobility or physical impairments by reducing the need for caregivers and increasing self-independence.
}\end{abstract}

\todo{To me, the abstract seems too detailed. It should be used to give quick information on your work. - Nicola}

\todo{``resource-efficient'': check hardware usage, I think I have the data, but I must add them in the Appendix just to be sure.}

\chapter{Introduction}\label{ch:introduction}
% add references to the unity env used for testing (perhaps)
% Start from the problem AKA controlling devices and current QoL for people with reduced mobility or physical disabilities (e.g. missing arms, legs, etc.)\\
% $\xrightarrow{\text{introduce}}$ HCI and present examples\\
% $\xrightarrow{\text{discussion leads to}}$ BCI\\
% $\xrightarrow{\text{finally move to}}$ EEG-based BCI\\
% $\xrightarrow{\text{introduce}}$ EEG-based BCI techniques, such as SSVEP, P300, and MI\\
% $\xrightarrow{\text{focus on}}$ MI (explaining why)\\
% $\xrightarrow{\text{introduce}}$ MI-based BCI applications and eventually videogames\\
% $\xrightarrow{\text{introduce}}$ MI-based BCI applications in real-world devices (e.g. robotic wheelchairs or prosthesis).\\

% Now introduce the problem and the research question: ``Is it possible to perform EEG-Motor Imagery classification in real-time?''\\

% $\xrightarrow{\text{introduce}}$ the research objectives and the hypotheses\\
% $\xrightarrow{\text{introduce}}$ solution (not detailed!)\\

% $\xrightarrow{\text{introduce}}$ the structure of the thesis
Even if nowadays there are rapid changes in the technological landscape, enhancing the quality of life (QoL) for individuals with reduced mobility or physical disabilities remains a critical challenge. 
People with physical impairments often face significant barriers in performing everyday tasks that many take for granted. 
Controlling devices autonomously can be particularly important for these individuals, allowing them to reduce the need for caregivers and assistive technologies.
The quest to empower these individuals through innovative solutions has led to significant research in Human-Computer Interaction (HCI).
HCI focuses on creating effective interfaces between users and computers, aiming to enhance the usability and accessibility of digital devices. 
Various HCI applications have demonstrated considerable promise in aiding individuals with disabilities. 
For example, speech recognition systems assist elderly people in everyday tasks~\cite{10444265}, while eye-tracking technologies enable assistive devices control through eye movements~\cite{10560196}. 
However, these methods often have limitations, such as requiring specific environmental conditions~\cite{khazaleh2024investigation} or extensive training periods or datasets~\cite{ke2024using}, and may not be suitable for all users.

While HCI technologies continue to evolve, the field has seen a growing interest in Brain-Computer Interfaces (BCIs)~\cite{yadav2020comprehensive}, which offer a direct communication pathway between the brain and external devices. 
BCIs have the potential to change the way individuals with severe disabilities interact with their environment by bypassing traditional physical limitations and using brain signals to control devices.
Among the various BCI techniques, Electroencephalography (EEG)-based BCIs are particularly promising. 
EEG is a non-invasive method that records electrical activity of the brain through electrodes placed on the scalp. 
It provides a relatively accessible and affordable means of capturing brain signals for BCI applications. 
Several EEG-based BCI paradigms have been developed, including Steady-State Visual Evoked Potential (SSVEP)~\cite{BCI_SSVEP_P300}, P300~\cite{BCI_SSVEP_P300}, and Motor Imagery (MI).

The latter involves the mental simulation of movement without any actual physical execution, such as imagining moving a limb. 
MI-based BCIs leverage this mental process to generate distinct EEG patterns that can be used to control external devices. 
MI is particularly appealing because it taps into the natural motor processes of the brain, making it intuitive and potentially more effective for users with motor impairments. 
Additionally, MI does not require external stimuli like SSVEP or P300, which can be advantageous in various practical applications.
MI-based BCI systems have been applied in a range of areas, from assisting the users in playing simple video games~\cite{10174453} to operating real-world devices like robotic wheelchairs~\cite{palumbo2021motor} and prosthetic limbs~\cite{10453986}.
These applications not only enhance entertainment and engagement but also provide significant functional benefits, improving independence and QoL for users.
Unfortunately, the translation of such systems to practical applications often involves challenges related to real-time processing and control.

Given the potential of MI-based BCIs, a critical question arises: ``\emph{Is it possible to perform EEG-Motor Imagery classification in real-time?}'' 
This question addresses the feasibility of using MI-BCI systems in practical, real-world scenarios where timely and accurate response is crucial.

The primary objectives of this research are to investigate the real-time classification of EEG signals during motor imagery tasks and to develop a robust system capable of controlling external devices based on these classifications. 

The hypotheses guiding this research include:
\begin{itemize}
    \item[\textbf{H1:}] \emph{It is feasible to accurately classify motor imagery tasks in quasi-real-time using EEG signals}.
    \item[\textbf{H2:}] \emph{A quasi-real-time MI-BCI system can be effectively integrated with external devices for practical applications}.
\end{itemize}

The first hypothesis addresses the technical feasibility of real-time EEG signal processing for MI classification, whitout this assumption the second hypothesis would not be possible to verify.
The second hypothesis focuses on the practical integration of the MI-BCI system with external devices, which is essential for real-world applications.

This thesis proposes a quasi-real-time EEG-MI categorization system that can connect with a range of assistive devices.
The approach attempts to use advanced signal processing and deep learning techniques to achieve high accuracy and low latency in EEG signal categorisation. 
We take advantage of techniques like models based on Long Short Term Memory~(LSTM)~\cite{hochreiter1997long}, to classify the user's motor intention, and data augmentation techniques, in particular, Generative Adversarial Networks~\cite{goodfellow2014generative}, to study the EEG signals and generate realistic samples.
Moreover, we leverage WebSocket~\cite{fette2011rfc} communications to interact with external devices and applications in real time or with a low delay between the signal classification and its application.
In this way, we aim to progress MI-based BCIs and, in the process, improve the quality of life (QoL) for people with physical limitations by answering the research issue and accomplishing the specified goals.


The remainder of this thesis is organized as follows, in Chapter~\ref{ch:related_works}, we discuss the background and related work in the fields of BCI, EEG signal processing, and MI-based applications.
%In Chapter \ref{ch:preliminary_work}, we describe the methods used for EEG data acquisition, signal processing, and classification.
%Maybe change chapters, the approach becomes technical setup, describes the pipeline and the data-related part and design idea for the game and the reason (also how I modelled the movement imagination into controls), methodology describes 
In Chapter~\ref{ch:approach}, we discuss the approach taken to solve the problem of classifying different types of motor imagery EEG signals in real time, including data collection, data augmentation, classification model, and virtual environment development.
In Chapter~\ref{ch:methodology}, we detail the development of the real-time MI-BCI system. % TODO split chapter in 4 (data preprocessing, network development, sample data validation, real data validation)
% In Chapter \ref{ch:results}, we present the experimental setup, results, and analysis of the system's performance.
% TODO maybe merge discussion and conclusion
In Chapter~\ref{ch:testing}, we discuss the results of the system's performance evaluation and the implications for real-world applications.
In Chapter~\ref{ch:human_study}, we present our idea of a pilot human study we wanted to conduct to evaluate the system's usability and user experience.
% In Chapter \ref{ch:schedule}, we outline the timeline for the completion of the research and the remaining tasks.
In Chapter~\ref{ch:conclusions}, we summarize the contributions of the thesis and outline directions for future research.

Apart from the main chapters, we provide and appendix with additional information on the used vocabulary (Appendix~\ref{app:vocabulary}).

\chapter{State of the Art}\label{ch:state_of_the_art}
\chapter{Preliminary Work}\label{ch:preliminary_work}
\section{Design}\label{sec:design}
To proceed with the implementation of the system, a design phase is necessary. 
This phase is crucial to the success of the project, as it will provide a clear understanding of the system's architecture and the technologies that will be used. 
This section will present the design of the system, including the architecture, the technologies, and the main challenges that will be faced during the implementation.

\subsection{System Architecture}
The system will be composed of two main components: the single-player game and the neural network that classifies the player's movement intentions.
The game will be developed using the Unity game engine, and the neural network will be implemented using either the TensorFlow or the SciKit-Learn library.
During the development of this project, several networks might be created, and the best one will be chosen to be integrated into the game.
The main reason for this is related to the investigation of the best neural network architecture for the classification of the player's movement intentions.
If the custom architecture does not perform well, the project will use a state of the art model, or the most suitable model found during the investigation.

The created networks will be hybrid models, combining convolutional and/or recurrent layers with other types of layers, such as graph convolutional layers and fully connected layers.
The input of the network will be the EEG recordings from the PhysioNet dataset, and the output will be the classification of the player's movement intentions.
The network will be trained using the EEG recordings and the corresponding labels from the dataset.
The trained network will be integrated into the game, and the player's EEG recordings will be used as input to the network to classify the player's movement intentions.

The EEG recordings will be preprocessed before being used as input to the network.
The preprocessing will include the removal of noise and the normalization of the data, other preprocessing techniques, following or diverging from the literature, might be used to improve the network's performances.
Both the training, validation and testing data will be preprocessed using the same techniques to ensure that the network is not biased towards the training data.
The noise removal will be done using a bandpass filter, and the EEG related libraries, such as MNE-Python and Braindecode, will be used to normalise and further preprocess the data.

\subsection{Technologies}
The game will be developed using the Unity game engine, and the neural network will be implemented using either the TensorFlow or the SciKit-Learn library.
The Unity game engine was chosen because it is a powerful and flexible tool for the development of games and simulations.
It provides a wide range of features, such as physics, rendering, and audio, and it supports the development of games for multiple platforms, including mobile, desktop, and consoles.
The game will be developed using the C\# programming language, which is the primary language used in Unity.

The TensorFlow and SciKit libraries were chosen because they are powerful and flexible tools for the development of neural networks.
They provide a wide range of features, such as optimisation algorithms, and support for various types of layers, such as convolutional, recurrent, and graph convolutional layers.

The EEG related libraries, MNE-Python and Braindecode, were chosen because they provide a wide range of features for the preprocessing and analysis of EEG recordings, also, they are integrated with neural networks libraries, such as TensorFlow and SciKit-Learn, they can be integrated with other libraries, such as NumPy, Pandas and Matplotlib, to further preprocess and analyse the EEG recordings, and lastly, there are other libraries that are seamlessly integrated with the chosen ones, such as moabb, which provides a wide range of EEG datasets.

\subsection{Challenges}
The system will face several challenges during the implementation.
The main challenges are related to the development of the game and the implementation of the neural network.
The development of the game will require the creation of a realistic and immersive environment, where the user can understand with ease the game's mechanics and the controls.
The game will also need to be optimised to perform the user's thought actions in real-time, and the game's mechanics will need to be designed to be compatible with the user's thought actions.

The implementation of the neural network will require the creation of a custom architecture that can classify the player's movement intentions with high accuracy.
The network will need to be trained using the EEG recordings and the corresponding labels from the dataset, and the training process will need to be optimised to achieve the best possible performance.
Also, the network will have to use as low amount of data as possible, to avoid unwanted latencies and provide a high interactivity with the game.
The network will have to be optimised for real-time classification of the player's movement intentions, and the network's architecture will need to be designed to be compatible with the EEG recordings.

The integration of the neural network into the game will require the creation of a pipeline that can preprocess the live EEG recordings and use them as input to the network.
Both, the game and the neural network will need to be optimised to work together and provide a seamless experience for the user.

\subsection{Summary}
This section presented the design of the system, including the architecture, the technologies, and the main challenges that will be faced during the implementation.
The system will be composed of two main components: the single-player game and the neural network that classifies the player's movement intentions.
The game will be developed using the Unity game engine, and the neural network will be implemented using either the TensorFlow or the SciKit-Learn library.
The game will be optimised to perform the user's thought actions in real-time, and the game's mechanics will be designed to be compatible with the user's thought actions.
The network will be trained using the EEG recordings and the corresponding labels from the dataset, and the trained network will be integrated into the game to classify the player's movement intentions.
The EEG recordings will be preprocessed before being used as input to the network, and the game and the neural network will be optimised to work together and provide a seamless experience for the user.

\chapter{Methodology}\label{ch:methodology}

\chapter{Discussion}\label{ch:results}
\chapter{Conclusions}\label{ch:conclusions}
\appendix
\chapter{Appendix 1}\label{app:appendix1}
put data to confirm the abstract note
\printbibliography
\printindex
\chapter*{Acknowledgements}
First of all, I would like to thank myself for the hard work and dedication that I put into this thesis, for not giving up, and for the sacrifices that I made to achieve this goal.
I would also like to thank my family for their encouragement, support and help, especially during the difficult times, and for their patience and understanding.
I would like to thank my friends for their support, encouragement, for the good times we shared, and the endless chill nights that we spent together.
I would like to thank my supervisors, Prof. Dr.~Michele Loreti and Prof. Dr.~Hannes H\"ogni Vilj\'amson, and co-supervisors, Prof. Dr.~Michela Quadrini and Dr.~Nicola Del Giudice, for their guidance and support throughout my research.
I would like to thank the University of Camerino and Reykjav\'ik University for providing me with the opportunity to study and conduct research.
Lastly, I want to thank the music and all the substances that kept me awake and alive, and saved me from several burnout, during the university years, especially during the writing of this thesis.

\end{document}