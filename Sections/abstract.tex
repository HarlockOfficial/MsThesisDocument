\begin{abstract}{
    Nowadays, enhancing the quality of life for individuals with reduced mobility or physical disabilities is a critical challenge.
    In recent years, new technologies and techniques, such as Human-Computer Interfaces and, more specifically, Brain-Computer Interfaces have risen as a viable solution.
    The research has focused on developing applications and devices that allow people to complete everyday tasks autonomously or with minimal need from external caregivers.
    This thesis focuses on brain-computer interfaces and tries to solve one major problem of these kinds of technologies: the time gap between the user action intention and the actual application of it.
    
    In this work, we present a quasi-real-time ElectroEncephaloGraphy Motor Imagery classification system, a pipeline to integrate the classification system and seamlessly control external devices or applications, and a data augmentation system that allows rapid and efficient testing.
    We will also present a virtual environment that we used to test the pipeline.
    
    To verify the network and the pipeline, we tested the system with an EEG cap and volunteers from Reykjavik University.
    We found the network to be accurate and resource-efficient and the pipeline to be fast and flawless, with an average of 650 ms between the user movement intention and the application in the virtual environment.

    This thesis is an initial work and a feasibility study towards real-time brain-controlled devices and applications.
    It also has the potential to enhance the quality of life for people with reduced mobility or physical impairments by reducing the need for caregivers and increasing self-independence.
}\end{abstract}
\todo{``resource-efficient'': check hardware usage, I think I have the data, but I must add them in Appendix just to be sure}