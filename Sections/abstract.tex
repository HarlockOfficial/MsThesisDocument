\begin{abstract}{
    Nowadays, enhancing the quality of life for individuals with reduced mobility or physical disabilities is a crucial challenge.
    In recent years, new technologies and techniques, such as Human-Computer Interfaces and, more specifically, Brain-Computer Interfaces have emerged as viable solutions.
    The research has focused on developing applications and devices that allow people to complete everyday tasks autonomously or with minimal need for external caregivers.
    This thesis focuses on brain-computer interfaces and tries to solve one major problem of these kinds of technologies: the time gap between the user action intention and the actual application of it.
    
    We present a quasi-real-time ElectroEncephaloGraphy Motor Imagery classification system, a pipeline to integrate the classification system and seamlessly control external devices or applications, and a data augmentation system that allows rapid and efficient testing.
    We will also present a virtual environment that we used to test the pipeline.
    
    We tested the system with an EEG cap and volunteers to verify the network and the pipeline.
    We found the network to be accurate and resource-efficient, and the pipeline fast and flawless, with an average of 650~ms between the user movement intention and the application in the virtual environment.
    This is an improvement over the average values found during the literature review, which report the need for an EEG recording between 4 and 8 seconds that has to be processed, classified, and sent to the controlled device.
}\end{abstract}

\todo{``resource-efficient'': check hardware usage, I think I have the data, but I must add them in the Appendix just to be sure.}
