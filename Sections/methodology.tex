\chapter{Methodology}\label{ch:methodology}
% \cite{trafton2014blink} 13 ms necessary to recognise objects
Following the introduction of the previous chapter, this chapter will describe the methodology used to develop the system.
The chapter is divided into five sections.
The first section will describe the data collection process, the second section will describe the data augmentation, the third section will describe the classification model, the fourth section will describe the virtual environment, and the fifth section will describe the evaluation pipeline.
The chapter will conclude with a summary of the methodology used in the development of the system.

\section{Data Collection}
The data collection process is a crucial step in the development of a machine learning system.
The data collection process is used to gather the data necessary to train the machine learning model.
The data collection process is divided into two steps: the data collection and the data preprocessing.
The data collection step is used to gather the data necessary to train the machine learning model.
The data preprocessing step is used to clean the data and prepare it for training.
% TODO expand from here, enter in technicalities

\section{Data Augmentation}
Data augmentation is a technique used to increase the size of the training dataset.
Data augmentation is used to improve the performance of the machine learning model.
Data augmentation is used to generate new data points by applying transformations to the existing data points.
Data augmentation is used to improve the generalization of the machine learning model.
Data augmentation is used to reduce overfitting.
In this project, we decided to use data augmentation to generate new data points to test the performance and generalization capabilities of the machine learning model.
% TODO expand from here, enter in technicalities

\section{Classification Model}
The classification model is a machine learning or deep learning model used to classify the data points.
The classification model is used to predict the class to which the input data point belongs.
The classification model is used to assign a label to the input data point.
In this project, we decided to train a set of machine learning and deep learning models to classify the data points.
Some of these methods, including support vector machines, linear discriminant analysis and convolutional neural networks, have been used as baseline models.
While the proposed LSTM method has been used as the main model.
% TODO expand from here, enter in technicalities

\section{Virtual Environment}
The virtual environment is a computer-generated environment that simulates the real world or a specific scenario.
The virtual environment is used to test the performance and usability of the machine learning model, and to evaluate the user experience.
In this project, we developed two virtual environments: an infinite runner game where the user can jump or move to left or right, and a maze where the user can freely move their character.
The virtual environments were developed using the Unity game engine.
% TODO expand from here, enter in technicalities

\section{The Framework}
The framework was developed to integrate all the components of the system and to provide a user-friendly interface.
It works as a pipeline that connects the data augmentation or collection from the headset, classification model, and virtual environment.
The framework is plug-and-play, meaning that it can be easily extended to support new data augmentation or generation methods, classification models, virtual environments or real devices.
% TODO expand from here, enter in technicalities