\chapter{State of the Art}\label{ch:state_of_the_art}
In this chapter, we will present the main concepts and projects that inspired the development of this thesis. 
We will start by presenting rehabilitation approaches and technologies and the importance of the gamification of rehabilitation. 
We will also present the main concepts of the Brain-Computer Interface (BCI) and the most relevant projects that use BCI and more specifically Motor Imagery for rehabilitation and devices control. 
Lastly, we will present concepts of usability and game feel that are important for the development of a usable and engaging application.

\section{Rehabilitation Approaches and Technologies}
Rehabilitation is a process that aims to restore the physical, sensory, and mental abilities of individuals who have lost them due to injury, illness, or other health conditions.
There are several approaches to rehabilitation, and the choice of the most appropriate one depends on the patient's condition and the goals to be achieved.
The main approaches to rehabilitation are physical therapy, occupational therapy, and speech therapy.
Physical therapy aims to restore the patient's physical abilities, such as strength, flexibility, and coordination.
Occupational therapy aims to help the patient to perform daily activities, such as eating, dressing, and bathing.
Speech therapy aims to help the patient to improve their communication skills, such as speaking, listening, and understanding language.
There are also other approaches to rehabilitation, such as music therapy, art therapy, and animal-assisted therapy, which can be used to complement the traditional approaches.
Works like \cite{202306.0333, 5567156, 10.4108/icst.pervasivehealth.2014.255277, trombetta}, aim to further aid the rehabilitation process using AR and VR applications, exergames, and serious games.
In \cite{BOUKHENNOUFA2022103197}, we can also find a review of state of the art solutions that involve werable sensors and machine learning techniques.
In the last decades, the use of Brain Computer Interfaces (BCI) has been gaining popularity in the field of rehabilitation, as they allow the control of devices and applications using only the brain signals.
The work from Bai et al. \cite{bai_immediate_2020} analyzes short and long term effects of BCI-based rehabilitation systems and concludes that these interfaces are safe for patients with strokes and effective at improving upper extremity motor function.

\section{Brain-Computer Interfaces (BCI)}
In \cite{robinson2021emerging}, the authors review the current state of the art for BCI technologies used to control robots and for motor rehabilitation.
In their work, they present the main types of BCI, such as invasive, and non-invasive, and the most common signal acquisition methods, such as Electroencephalography (EEG) and Magnetoencephalography (MEG).
In \cite{altaheri_deep_2023}, Altaheri et al. present the EEG Motor Imagery as one of the ``most common BCI paradigms that have been used extensively in smart healthcare applications such as poststroke rehabilitation and mobile assistive robots''.
In this work, the authors also review existing applications, like \cite{tang2020motor} that presents an EEG MI system used to control a wheelchair, publicly available datasets for EEG Motor Imagery, such as the ones from the BCI Competition IV and the ones provided by PhysioNet\cite{goldberger2000physiobank, schalk2004bci2000}, and present the most common deep learning algorithms used for EEG Motor Imagery classification, such as Convolutional Neural Networks (CNN) \cite{lawhern2018eegnet}, Long Short-Term Memory (LSTM) and Transformer Based solutions \cite{sharma_deep_2023}.

\section{Usability and Game Feel}
Usability is a key factor in the development of applications, as it can influence the user's experience and satisfaction.
In \cite{swink2008game}, the author presents the main concepts of game feel in video games, such as responsiveness, smoothness, and consistency, and how they can be used to improve the user experience.
The thesis from Nguyen \cite{nguyen2012human}, shows that there is a weak correlation between Human Computer Interaction and Game Design, and suggests the use of heuristics and character automation to improve the overall user experience.
In \cite{doi:10.1080/10447318.2019.1612213}, the authors present a review of the main videogames developed for consumer grade EEG devices, and mention that only few research works have focused on the usability and qualitative aspects of these games.

\section{Main Differences}
Using these works and documents as a reference, we can see that there is a gap in the literature regarding the development of a BCI-based rehabilitation application that is both usable and engaging.
Most of the works focus on the technical aspects of the BCI, such as signal acquisition and processing, and the development of the rehabilitation exercises, but do not focus on the usability and game feel of the application.
In this thesis, we aim to fill this gap by developing a BCI-based application that is both usable and engaging, and by evaluating the user experience of the application using a mixed-methods approach.
Our project will use the BCI Motor Imagery paradigm to control the application, will not specifically focus on rehabilitation exercises, but will provide a set of mini-games curated to provide a good level of usability and a fun experience.
The games will be controlled by the intention of movements of hands and feet, and the user will be able to play them using only their brain signals.
