\chapter{Conclusions and Future Works}\label{ch:conclusions}
% review the contribution of this thesis
% new model/framework/some insights for bci in games

This thesis has made significant contributions to the field of brain-computer interfaces (BCIs), particularly in the realm of real-time EEG-based motor imagery classification systems. 
Our research primarily focused on addressing the critical time delay between the user's intent and the execution of control commands in BCI applications. 
By developing a quasi-real-time EEG Motor Imagery classification system, we aimed to enhance the responsiveness and accuracy of BCI systems, making them more practical for real-world applications.

Our research involved integrating the classification system into a seamless pipeline for controlling external devices or applications and implementing an innovative data augmentation system to facilitate rapid and efficient testing. 
The virtual environment designed for testing this pipeline demonstrated the efficacy of our approach, providing a controlled setting where we could evaluate the system's performance under various conditions.

One of the key achievements of this thesis is the development of a classification network that is both accurate and resource-efficient. 
The network's performance in real-time scenarios was reliable, ensuring that the system could be used effectively in practical applications. 
Additionally, the pipeline we developed exhibited an average delay of just 650 milliseconds between the user's movement intention and the corresponding action in the virtual environment. 
This is a substantial improvement over existing systems, which typically require 4 to 8 seconds for processing.
This reduction in latency is critical for making BCIs more user-friendly and responsive.

The innovative data augmentation techniques we employed, such as stochastic noise addition and GAN-generated data points, played a crucial role in enhancing the robustness and generalizability of our classification models. 
These techniques allowed us to create a more diverse and comprehensive dataset, improving the system's ability to handle variations in EEG signals across different users and sessions. 
This advancement is particularly important for ensuring that the system can perform well in real-world settings where EEG signals can be highly variable.

Despite these advancements, there are areas for future work. 
Conducting more extensive studies with a larger and more diverse participant pool will be crucial to further validate and refine the system.
Such studies will help us understand how the system performs across different populations and identify any potential limitations or areas for improvement. 
Additionally, exploring user-tailored data training and fine-tuning the network for individual subjects could enhance the system's performance in real-world applications. 
Personalizing the system to individual users' unique EEG patterns could significantly improve accuracy and responsiveness.

Another area for future research is the integration of the BCI system with more complex and varied external devices or applications. 
While our virtual environment provided a useful testing ground, real-world applications often involve interacting with a wide range of devices and systems. 
Future work should explore how our classification and control pipeline can be adapted to different contexts and integrated with other technologies, such as robotics or smart home systems.

In conclusion, this thesis has laid a solid foundation for real-time BCI systems, offering a promising pathway toward improving the autonomy and quality of life for individuals with reduced mobility or physical disabilities. 
Our research has demonstrated the feasibility and potential of quasi-real-time EEG Motor Imagery classification systems, and the advancements we have made in reducing latency and enhancing accuracy are significant steps forward. 
Future research and development efforts should focus on expanding the applicability and scalability of our system to ensure its practical deployment and widespread adoption. 
By continuing to refine and improve upon our work, we can move closer to realizing the full potential of BCIs to transform the lives of those who rely on these technologies.
