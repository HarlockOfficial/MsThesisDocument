\chapter{Pilot Human Study}\label{ch:human_study}
\todo{Change content after actual study, this is an optimistic base to include the actual data later}
% evaluation of the model in a realtime scenario with real people
% I really wanted to ``close the loop'' of the evaluation by including the human in it, because I was interested in the interaction between human and the system.
% Technical performances may pose some problems, but I want to get the whole picture.
% later it is possible to iterate over everything to improve.
% Study with eeg cap.
After the testing phase, we wanted to close the loop by conducting a pilot human study to evaluate the system in a real-time scenario with real people.
The goal of the study was to evaluate the interaction between the user and the system. 
We wanted to understand how the user perceived the system and how the system performed in a real-world scenario. 
The study was conducted with a small group of participants, and the results were used to improve the system and to plan a larger study in the future.

\section{Study Design}
The study was designed to evaluate the system in a real-time scenario with real people.
The participants were asked to play a simple video game using the system.
The game was designed to be simple and easy to play, to avoid any bias due to the complexity of the game.
The participants were asked to wear an EEG cap and to perform motor imagery tasks to control the game.
The study was conducted in a controlled environment, to avoid any external interference.
The participants were asked to fill out a survey after the study, to evaluate the system and their experience.

\section{Participants}
We recruited a small group of participants for the study.
The participants were all healthy adults, with no history of neurological disorders.
The participants were all informed about the study and gave their consent to participate.

\section{Procedure}
The study was conducted in a controlled environment.
The participants were asked to wear an EEG cap and to perform motor imagery tasks to control the game.
The participants were asked to play the game for a fixed amount of time, and then to fill out a survey to evaluate the system and their experience.
% Note: not sure of the following sentence
The study was conducted in a single session, and the participants were free to ask questions and to take breaks if needed.

\section{Game Design}
The game was designed to be simple and easy to play.
The game was a 3D walking simulator, where the player controlled a character in a virtual environment.
The player could control the character by performing motor imagery tasks, such as imagining moving their left hand to move the character left, and imagining moving their right hand to move the character right.
The game was designed to be challenging but not frustrating, to keep the participants engaged.

\section{Data Collection}
We collected data during the study to evaluate the system and the user experience.
We collected EEG data to evaluate the system performance, and we collected survey data to evaluate the user experience.
The data was analyzed to identify any issues with the system and to improve the system for future studies.

\section{User Experience Survey}
We used a Likert Scale~\cite{likert1932technique} to evaluate the user experience.
The participants were asked to rate their experience with the system on a scale from 1 to 10, where 1 was very bad and 10 was very good.
The participants were also asked to provide feedback on the system and to suggest improvements.
The questions were designed to evaluate the system performance and the user experience.
And followed the System Usability Scale (SUS)~\cite{brooke1996sus} guidelines and included many questions from the Game Experience Questionnaire~\cite{ijsselsteijn2013game}.

\section{Results}
The results of the study showed that the system performed well in a real-time scenario with real people.
The participants were able to control the game using the system, and they reported a positive experience.
The participants also provided feedback on the system, which was used to improve the system for future studies.

\section{Pilot Study Discussion}
The pilot human study was successful in evaluating the system in a real-time scenario with real people.
The study showed that the system performed well, and the participants reported a positive experience.
The study also identified some issues with the system, which were used to improve the system for future studies.
The results of the study will be used to plan a larger study in the future.
